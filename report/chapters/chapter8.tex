\chapter{Μελλοντικές επεκτάσεις}
\label{chapter:future_work}


% Έλεγχος εύρεσης πορτών σε πραγματικά ogm (επόμενο βήμα η μελέτη κανονικών - πραγματικών κόσμων)
% amcl θελει δουλιτσα γιατι στα πειραματα κάλυψης διαπιστώνεται ότι συχνά χάνει το λοκαλιζατιον του.
% Πραγματικά πειράματα πλοήγησης (φουλ κοβερατζ)
% πολλαπλά ρομπότ case
% δυναμική επεξεργασία αλληλουχίας στα δωμάτια, χρηση αρχικης θέσης, μελέτη αλληλοκάλυψης δωματίων.
% κατηγοριοποιηση δωματιων - διαφορετικη συμπεριφορα σε καθενα απ αυτα

Αρχικά, μια σημαντική βελτίωση του συστήματος που δεν μελετήθηκε στην παρούσα διπλωματική είναι η βελτίωση του συστήματος εντοπισμού της θέσης του οχήματος, δηλαδή του AMCL. Από το σύνολο των πειραμάτων παρατηρήθηκε ότι το όχημα αποπροσανατολίζεται σε περιπτώσεις πλοήγησης μεγάλης χρονικής διάρκειας και σε σύνθετους χώρους που η απαιτούμενη υπολογιστική ισχύς είναι αρκετά αυξημένη. Ο ακριβής εντοπισμός της θέσης του είναι καίριας σημασίας, διότι τυχόν αποκλίσεις θα οδηγήσουν σε λανθασμένες εκτιμήσεις της θέσης των προϊόντων, κάτι που πρέπει να αποφευχθεί.

Επιπλέον, είναι σημαντικό η υπάρχουσα υλοποίηση να εξεταστεί και σε πραγματικές καταστάσεις για να ελεχθεί κατά πόσο η διαδικασία μπορεί να χρησιμοποιηθεί σε κανονικές συνθήκες. Οι προσομοιώσεις μπορεί να προσεγγίζουν ικανοποιητικά την πραγματικότητα, ωστόσο υπάρχουν λεπτομέρειες που μπορούν να επηρεάσουν σημαντικά. Τα OGM που προκύπτουν από διαδικασία SLAM πραγματικού χώρου δεν είναι τόσο ακριβή όσο τα OGM από προσομοιώσεις, ενώ η πλοήγηση ενός πραγματικού οχήματος είναι πιθανό να επιφέρει αποκλίσεις στα αποτελέσματα τα οποία δεν μπορούν να προβλεφθούν. Επομένως, πρέπει να ελεχθούν τα προβλήματα ή οι αστοχίες που είναι πιθανό να προκύψουν στην τοπολογική ανάλυση και στην πλήρη κάλυψη ενός πραγματικού χώρου.

Μια ακόμη επέκταση του συστήματος είναι η χρήση πολλαπλών ρομποτικών οχημάτων. Η κατανομή του χώρου κάλυψης σε περισσότερους πράκτορες, η επικοινωνία μεταξύ τους και ο συγχρονισμός τους απαιτεί μια πιο περίπλοκη υλοποίηση, ωστόσο επιφέρει σημαντική βελτίωση στον χρόνο πλήρους κάλυψης του χώρου και πιθανώς και στην ακρίβεια εντοπισμού των προϊόντων.

Ο εντοπισμός δωματίων μπορεί να αξιοποιηθεί ακόμη καλύτερα με την ταξινόμηση τους σε διάφορες κατηγορίες, όπως για παράδειγμα σε δωμάτια και διαδρόμους. Μια μέθοδος ταξινόμησης υλοποιήθηκε στο \ref{section:room_classification} και μπορεί να χρησιμοποιηθεί σε περιπτώσεις μη ομοιόμορφης κατανομής των προϊόντων στους διάφορους χώρους. Σε ένα ρεαλιστικό παράδειγμα αποθήκης είναι πιθανό προϊόντα να έχουν τοποθετηθεί μόνο στους διαδρόμους. Επομένως, το όχημα θα πρέπει να προσαρμόσει τη συμπεριφορά του στην εκάστοτε κατάσταση. Η υλοποίηση ενός τέτοιου πλήρως προσαρμοστικού συστήματος θα γλυτώσει το όχημα από περιττές διαδρομές και θα εξοικονομίσει χρόνο και ενέργεια.

Τέλος, όσον αφορά το μονοπάτι σημείων κάλυψης υπάρχουν ορισμένες επεκτάσεις που μπορούν να μελετηθούν. Μια σημαντική και σύνθετη μέθοδος αποτελεί η υλοποίηση μιας δυναμικής αναπροσαρμογής της αλληλουχίας σημείων κάθε δωματίου. Στη μέθοδο που αναπτύχθηκε στο \autoref{section:room_path_implementation} η σειρά προσπέλασης των σημείων έχει εξαρχής οριστεί, ενώ η σειρά επίσκεψης των δωματίων προσαρμόζεται στο δωμάτιο που βρίσκεται το όχημα κατά την εκκίνηση της διαδικασίας. Αντίστοιχη προσαρμογή θα μπορούσε να μελετηθεί και για τα σημεία κάθε δωματίου. Η αρχική θέση είναι ένα στοιχείο που θα μπορούσε να χρησιμοποιηθεί για να βελτιώσει το συνολικό μήκος κίνησης σε πραγματικό χρόνο. Επίσης, ένα τέτοιο σύστημα θα μπορούσε να αποφασίζει να αγνοήσει σημεία ή χώρους οι οποίοι έχουν ήδη καλυφθεί, μειώνοντας το συνολικό χρόνο της διαδικασίας. Μια άλλη μέθοδος είναι η αντιμετώπιση του μονοπατιού και ως ένα ενιαίο, αντί αποκλειστικά ανά δωμάτιο. Με τον τρόπο αυτό μπορεί ένα κεντρικό δωμάτιο να καλύπτεται τμηματικά και ενδιάμεσα του να παρεμβάλλονται άλλα δωμάτια, μειώνοντας το συνολικό μήκος της διαδρομής.
