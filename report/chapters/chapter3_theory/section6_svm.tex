\section{Support Vector Machines}
\label{section:svm}

Στον κλάδο της μηχανικής μάθησης υπάρχουν πολλοί διαφορετικοί αλγόριθμοι δημιουργίας μοντέλων πρόβλεψης. Μια κατηγορία τέτοιων μοντέλων είναι τα SVM, τα οποία χρησιμοποιούνται στην επιβλεπόμενη μάθηση και περιέχουν αλγορίθμους ανάλυσης των δεδομένων που αφορούν προβλήματα ταξινόμησης και παλινδρόμησης. Το σύνολο δεδομένων αποτελείται από δείγματα που περιέχουν τιμές από ορισμένες μεταβλητές, τα χαρακτηριστικά τους, και μια κλάση στόχο, στην οποία αντιστοιχεί το κάθε δείγμα. Συνήθως χρησιμοποιούνται σε δυαδικά προβλήματα ταξινόμησης, όπου η κλάση στόχος παίρνει τις τιμές ΝΑΙ και ΟΧΙ ή 1 και 0. Στόχος των μοντέλων αυτών είναι να προβλέψουν την κλάση ενός νέου άγνωστου δείγματος, χρησιμοποιώντας μόνο τις τιμές από τα χαρακτηριστικά του. Ένα τέτοιο παράδειγμα παρουσιάζεται στο \ref{fig:svm_example}, όπου και τα διάφορα δείγματα περιέχουν δύο χαρακτηριστικά και διαχωρίζονται από το μοντέλο με την πράσινη ευθεία - όριο σε δύο χώρους, ανάλογα με την κλάση τους. Οι δύο κλάσεις για οπτικούς λόγους εμφανίζονται με διαφορετικό χρώμα σε κάθε δείγμα.

Τα μοντέλα SVM προσπαθούν να συσχετίσουν τις τιμές των δειγμάτων με τις κλάσεις τους, αναλύωντας τον χώρο τιμών των διάφορων μεταβλητών. Συμπεριφέρονται στα δεδομένα σαν σημεία στον Ν-διάστατο χώρο και προσπαθούν να διαχωρίσουν τον χώρο αυτό σε δύο χώρους, έναν για κάθε κλάση, βρίσκοντας το βέλτιστο υπερεπίπεδο διαχωρισμού. Έτσι, κάθε νέο δείγμα που εμφανίζεται, εισέρχεται στον χώρο αυτό και το μοντέλο ελέγχει σε ποιον από τους δύο χώρους ανήκει, στον οποίο και ταξινομείται.


\begin{figure}
    \centering
    \includegraphics[width=0.5\textwidth]{./images/chapter3/svm_example.png}
    \caption{Διαχωρισμός χώρου με μοντέλο SVM}
    Πηγή: \href{https://eight2late.wordpress.com/2017/02/07/a-gentle-introduction-to-support-vector-machines-using-r/}{https://eight2late.wordpress.com/2017/02/07/a-gentle-introduction-to-support-vector-machines-using-r/}
    \label{fig:svm_example}
\end{figure}

Ο διαχωρισμός του Ν-διάστατου χώρου γίνεται γραμμικά, ωστόσο υπάρχουν μέθοδοι προσέγγισης και μη γραμμικών προβλημάτων με SVM μοντέλα χρησιμοποιώντας τους λεγόμενους πυρήνες. Οι πυρήνες αυτοί μετασχηματίζουν τα δεδομένα με τέτοιον τρόπο, ώστε το μετασχηματισμένο σύνολο τιμών να είναι γραμμικώς διαχωρίσιμο.

Στην παρούσα διπλωματική εργασία γίνεται χρήση ενός γραμμικού SVM μοντέλου για την ταξινόμηση του κάθε χώρου σε δωμάτιο ή διάδρομο.

