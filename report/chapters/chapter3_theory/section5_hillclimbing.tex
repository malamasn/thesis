\section{Αλγόριθμός Hill Climbing}
\label{section:hill_climbing}

O αλγόριθμος \emph{ανάβασης πλαγιάς} (Hill Climbing - HC)\footnote{\href{https://en.wikipedia.org/wiki/Hill_climbing}{https://en.wikipedia.org/wiki/Hill\_climbing}} είναι μια τεχνική βελτιστοποίησης που ανήκει στην κατηγορία της τοπικής αναζήτησης της αριθμητικής ανάλυσης. Είναι ένας επαναληπτικός αλγόριθμος που ξεκινάει με μία αυθαίρετη λύση στο πρόβλημα που μελετάται και προσπαθεί να βρει συνεχώς μια καλύτερη αλλάζοντας στοιχειωδώς την λύση. Εαν κάποια αλλαγή οδηγεί σε καλύτερο αποτέλεσμα, τότε η λύση αυτή κρατείται ως η τρέχουσα βέλτιστη και η διαδικασία συνεχίζεται με μια επόμενη αλλαγή. Η διαδικασία τερματίζεται όταν σταματήσει να υπάρχει κάποια αλλαγή που να βελτιώνει την τρέχουσα λύση. 

Ο αλγόριθμος αυτός βρίσκει την καλύτερη δυνατή λύση του προβλήματος μέσα σε μια τοπική περιοχή, που εξαρτάται αποκλειστικά από την αρχική αυθαίρετη λύση. Αυτό σημαίνει ότι ο αλγόριθμος μπορεί να εγκλωβιστεί σε τοπικά μέγιστα, δίχως να μπορεί να προσεγγίσει τα ολικά μέγιστα. Ωστόσο, το γεγονός ότι είναι πολύ απλός και γρήγορος στην υλοποίηση του τον καθιστούν χρήσιμο σε προβλήματα που ο χρόνος υπολογισμού είναι σημαντικός ή σε προβλήματα που η ολική λύση δεν απέχει σημαντικά από τις τοπικές.

Ταυτόχρονα, έχουν δημιουργηθεί διάφορες παραλλαγές του, με στόχο την αποφυγή του σημαντικού αυτού μειονεκτήματος που έχει. Δύο τέτοιες παραλλαγές χρησιμοποιήθηκαν σε αυτή την διπλωματική εργασία. Στην πρώτη παραλλαγή χρησιμοποιούνται επανεκκινήσεις της διαδικασίας όταν αυτή κολλάει σε τοπικά μέγιστα (Random Restart Hill Climbing - RRHC)\footnote{\href{https://en.wikipedia.org/wiki/Hill_climbing#Variants}{https://en.wikipedia.org/wiki/Hill\_climbing#Variants}}. Πιο αναλυτικά, εκτελείται ο απλός αλγόριθμος hill climbing μέχρι να φτάσει σε ένα τοπικό μέγιστο, όπου και η λύση αυτή κρατείται. Στη συνέχεια, εκτελείται ξανά ο hill climbing με διαφορετική αρχική λύση. Έτσι, εαν το νέο τοπικό μέγιστο αποτελεί βελτίωση του προηγούμενου, το αντικαθιστά. Η διαδικασία αυτή επαναλαμβάνεται για έναν συγκεκριμένο αριθμό επαναλήψεων που έχουν ορισθεί στην αρχή της διαδικασίας. Η δεύτερη παραλλαγή χρησιμοποιεί μια στοχαστική πιθανότητα για να αποδέχεται και αλλαγές που δεν βελτιώνουν την τρέχουσα λύση. Αυτή δίνει την δυνατότητα αποφυγής ορισμένων τοπικών ακρότατων, καθώς η λύση δεν περιορίζεται σε μια μικρή περιοχή τιμών. Μάλιστα, η πιθανότητα αυτή συνήθως με το πέρασμα των επαναλήψεων τείνει να μεταβάλλεται με τρόπο τέτοιο, ώστε ο αλγόριθμος να δέχεται όλο και περισσότερες αλλαγές που βελτιώνουν αυστηρά την λύση, προσεγγίζοντας σε άπειρο χρόνο την αρχική εκδοχή του hill climbing.

