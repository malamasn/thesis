\chapter{Ταξινόμηση Δωματίων}
\label{section:room_classification}

Στην ενότητα \ref{section:map_annotation} υλοποιήθηκε η μελέτη ενός γνωστού χώρου και ο διαχωρισμός του στα επιμέρους δωμάτια. Κάτι που δεν μελετήθηκε στον κορμό της διπλωματικής αυτής, αλλά φέρει επιστημονικού ενδιαφέροντος, είναι η ταξινόμηση των επιμέρους χώρων ανάλογα με το είδος τους. Ο διαχωρισμός που μελετήθηκε περιέχει δύο κλάσεις, τους διαδρόμους και τα δωμάτια. Αυτός μπορεί να χρησιμοποιηθεί σε περιπτώσεις που το ρομποτικό όχημα είναι επιθυμητό να έχει διαφορετική συμπεριφορά ανάλογα με τον χώρο στον οποίο βρίσκεται. Για παράδειγμα, μπορεί να είναι γνωστό ότι δεν υπάρχουν προϊόντα στους διαδρόμους, επομένως δεν μελετάται η κάλυψη των τοίχων των διαδρόμων. 

Στόχος είναι ο χαρακτηρισμός κάθε χώρου χρησιμοποιώντας τα λιγότερα δυνατα στοιχεία. Η μέθοδος που υλοποιήθηκε εξάγει στοιχεία από το ήδη υπολογισμένο GVD, το οποίο άλλωστε περιέχει την δομή του χώρου. Μετά την διαδικασία διαχωρισμού των κόμβων του χώρου σε δωμάτια \ref{subsection:find_room_nodes}, κάθε χώρος αποτελείται από τους κόμβους με έναν, τρεις ή περισσότερους γείτονες στο GVD όπως είχαν εντοπιστεί εξ αρχής, καθώς και από τους κόμβους με δύο γείτονες που όμως δεν αποτελούν σημεία πόρτας. Εξάγονται, λοιπόν, ορισμένα χαρακτηριστικά από αυτούς τους κόμβους τα οποία είναι τα εξής:

\begin{itemize}
    \setlength\itemsep{-0.2em}
    \item αριθμός κόμβων του δωματίου
    \item μέσος όρος αριθμού γειτόνων των κόμβων
    \item μέση απόσταση μεταξύ των κόμβων μετασχηματισμένη σε μέτρα
    \item μέση τιμή της τιμής brushfire των κόμβων μετασχηματισμένη σε μέτρα
    \item διακύμανση της τιμής brushfire των κόμβων μετασχηματισμένη σε μέτρα
    \item μέση τιμή της απόσταση κάθε κόμβου με τον κοντινότερο του μετασχηματισμένη σε μέτρα
\end{itemize}

Σημειώνεται ότι οι τιμές brushfire είναι υπολογισμένες σε pixel του OGM. Η απόσταση ενός pixel αντιστοιχεί σε μια σταθερή πραγματική απόσταση σε κάθε χάρτη, που ονομάζεται ανάλυση (resolution) του χάρτη και προκύπτει κατά το SLAM του χώρου. Έτσι, τα δεδομένα μετασχηματίζονται σε πραγματικές αποστάσεις, ώστε να είναι γενικευμένα και να μην παρουσιάζουν πρόβλημα ακόμη και για τον ίδιο χώρο με διαφορετική ανάλυση.

Συλλέχθηκαν 492 δείγματα από 70 χάρτες του \cite{li2019houseexpo} για τον σχηματισμό ενός συνόλου δεδομένων. Για την εκπαίδευση του μοντέλου έγινε χρήση ενός γραμμικού SVM\footnote{href{https://scikit-learn.org/stable/modules/svm.html}{https://scikit-learn.org/stable/modules/svm.html}}. Η επιλογή αυτή έγινε, καθώς όπως φάνηκε από τα αποτελέσματα τα δεδομένα είναι γραμμικώς διαχωρίσιμα. Το μοντέλο χωρίστηκε σε 70\% - 30\% αντίστοιχα για το σύνολο εκπαίδευσης και το σύνολο ελέγχου. Σε 10 πειράματα εκπαίδευσης το αποτέλεσμα είναι ποσοστό ακρίβειας 89,523 \% κατά μέσο όρο, με το μικρότερο ποσοστό να βρίσκεται στο 85,13 \% και το υψηλότερο στο 93,24 \%.

Το συμπέρασμα της μελέτης είναι ότι κάθε χώρος μπορεί να ταξινομηθεί με μεγάλη ακρίβεια σε δωμάτιο ή διάδρομο εξάγοντας δεδομένα από τα σημαντικά σημεία του GVD. Η μέθοδος αυτή είναι αρκετά απλή στην υλοποίηση και γρήγορη στην εκτέλεση, καθώς χρησιμοποιεί στοιχεία που είναι ήδη γνωστά. Τέλος, οι χώροι είναι γραμμικώς διαχωρίσιμοι χάρη στην χρήση της μέσης τιμής και της διακύμανσης των τιμών brushfire στους κόμβους κάθε δωματίου. Τα δύο αυτά χαρακτηριστικά αποτελούν τα πλέον καθοριστικά, διότι η μορφή ενός διαδρόμου είναι αρκετά ομοιόμορφη. Αυτό έχει ως αποτέλεσμα μικρές τιμές διακύμανσης των brushfire, αλλά και μικρές τιμές brushfire, συγκριτικά με τα δωμάτια που έχουν συνήθως πιο σύνθετη δομή.
