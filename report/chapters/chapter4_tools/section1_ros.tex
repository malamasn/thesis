\section{Robot Operating System (ROS)}
\label{section}

To Robot Operating System \cite{ros2009} αποτελεί το πιο διαδεδομένο σύστημα υλοποίησης ρομποτικών συστημάτων. Αποτελεί ένα μεσολειτουργικό (middleware) σύστημα, το οποίο διασυνδέει το λογισμικό (software) με το υλικό (hardware) με έναν τέτοιο τρόπο που η δημιουργία ρομποτικών συστημάτων και εφαρμογών είναι πιο απλή και γρήγορη. Περιλαμβάνει πολλά χαρακτηριστικά των λειτουργικών συστημάτων, όπως ο χαμηλού επιπέδου έλεγχος των συσκευών, η δημιουργία πλήθους διεργασιών και η παραλληλοποίηση των συστημάτων, η μετάδοση μηνυμάτων μεταξύ των διεργασιών αυτών, η διαχείρηση πακέτων. Επίσης, παρέχει χρήσιμα εργαλεία και βιβλιοθήκες για την ολοκληρωτική διαχείρηση κώδικα από πολλαπλούς υπολογιστές. 

Το ROS είναι μια πλατφόρμα ανοικτού κώδικα που έχει ως κύριο στόχο την απλοποίηση της διαδικασίας δημιουργίας ρομποτικών εφαρμογών. Η δόμηση της είναι τέτοια που εξυπηρετεί αυτό στον σκοπό. Συγκεκριμένα, ο κώδικας μπορεί και είναι επαναχρησιμοποιήσιμος, καθώς η κάθε εφαρμογή αποτελείται από ένα πλήθος κατανεμημένων διεργασιών (Nodes) που δίνει την δυνατότητα ανεξάρτητης ανάπτυξης των διαφόρων διεργασιών. Οι διεργασίες ομαδοποιούνται σε πακέτα (Packages) και στοίβες (Stacks), ώστε να είναι εύκολη η διανομή και ο έλεγχος τους μεταξύ των ερευνητών. Τέλος, μόνο κατά την συνολική εκτέλεση τους υπάρχει η ανάγκη συγκέντρωσης των διάφορων αυτών τμημάτων.
