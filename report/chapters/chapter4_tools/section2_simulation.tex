\section{Εργαλεία Προσομοίωσης}
\label{section:simulation_tools}

Το Gazebo\footnote{\href{http://gazebosim.org/}{http://gazebosim.org/}} αποτελεί ένα περιβάλλον προσομοίωσης ρομποτικών εφαρμογών σε τρισδιάστατα εικονικά περιβάλλοντα. Περιλαμβάνει όλους τους φυσικούς νόμους ενός πραγματικού περιβάλλοντος, κάτι που έχει ως αποτέλεσμα την προσομοίωση πραγματικών πειραμάτων σε ρεαλιστικές συνθήκες με μικρά σφάλματα σε σχέση με τα αντίστοιχα πραγματικά πειράματα. Επιπλέον, περιλαμβάνει ένα μεγάλο σύνολο από διαθέσιμα ρομπότ, αισθητήρες και περιβάλλοντα, κάνοντας πιο εύκολη την υλοποίηση των πειραμάτων.

Το ρομποτικό όχημα που χρησιμοποιείται στην εργασία αυτή είναι το turtlebot2\footnote{\href{https://www.turtlebot.com/turtlebot2/}{https://www.turtlebot.com/turtlebot2/}} το οποίο είναι προμηθευμένο με ένα LiDAR τοποθετημένο στο πάνω μέρος του για την ασφαλή κίνηση και την ακριβή εκτίμηση της θέσης του στον χώρο.

Επίσης, σημειώνεται το ROS έχει την υποδομή της άμεσης σύνδεσης των υπολοίπων συστημάτων με το Gazebo, καλώντας το κατά την εκκίνηση του συνόλου των διεργασιών. Μάλιστα, το Gazebo λειτουργεί ως μια ανεξάρτητη διεργασία, η οποία επικοινωνεί συνεχώς με όλες τις υπόλοιπες. 

Το ROS visualiziation ή Rviz\footnote{\href{http://wiki.ros.org/rviz}{http://wiki.ros.org/rviz}} χρησιμοποιείται για την τρισδιάστατη οπτικοποίηση όλων των δεδεμένων του συστήματος το οποίο τρέχει στον υπολογιστή. Αυτά τα δεδομένα μπορεί να αφορούν μετρήσεις από διάφορους αισθητήρες, πληροφορίες κατάστασης του ρομπότ και του περιβάλλοντα χώρου, ένα εικονικό μοντέλο του ρομπότ κ.α. Επίσης, μπορεί να παρουσιάσει πληροφορίες που προκύπτουν από τις διεργασίες που υλοποιούνται, όπως για παράδειγμα την επισήμανση του σημείου στόχου κατά την πλοήγηση του ρομπότ σε έναν χώρο.

