\section{Γλώσσες Προγραμματισμού}
\label{section:software}

Οι ρομποτικές εφαρμογές που χρησιμοποιούν την υποδομή του ROS συνήθως αναπτύσσονται είτε σε Python\footnote{\href{https://www.python.org/}{https://www.python.org/}} είτε σε C++\footnote{\href{http://www.cplusplus.com/}{http://www.cplusplus.com/}}. Σε αυτή την διπλωματική χρησιμοποιήθηκε η Python, καθώς δίνει τη δυνατότητα εύκολης ανάπτυξης κώδικα και γρήγορου ελέγχου. Έχει, όμως, ένα σημαντικό μειονέκτημα. Ο χρόνος εκτέλεσης κώδικα γραμμένο σε Python είναι μεγάλος. Για τον λόγο αυτό, σε ορισμένες περιπτώσεις που η ταχύτητα εκτέλεσης είναι μείζονος σημασίας, έχουν αναπτυχθεί αλγόριθμοι σε C και έχουν εισαχθεί στον κώδικα της Python με την χρήση της βιβλιοθήκης Cffi\footnote{\href{https://cffi.readthedocs.io/en/latest/}{https://cffi.readthedocs.io/en/latest/}}. Τέλος, μια από τις πιο σημαντικές βιβλιοθήκες που χρησιμοποιείται κατά κόρον είναι η \emph{numpy}\footnote{\href{https://www.numpy.org/}{https://www.numpy.org/}} η οποία διαχειρίζεται τις μεταβλητές των προγραμμάτων και υλοποιεί γρήγορα πράξεις μεταξύ τους.


