\chapter{Συμπεράσματα}
\label{chapter:conclusions}

Στο κεφάλαιο αυτό παρουσιάζονται συνοπτικά τα συμπεράσματα που προέκυψαν από τα αποτελέσματα του συνόλου των πειραμάτων. Στη συνέχεια αναφέρονται προβλήματα που παρουσιάστηκαν κατά τη διάρκεια των υλοποιήσεων και των πειραμάτων.


\section{Γενικά Συμπεράσματα}

Στα πλαίσια της παρούσας διπλωματικής εργασίας αναπτύχθηκε μια διαδικασία διαχωρισμού ενός χάρτη OGM στα επιμέρους δωμάτια του και ο υπολογισμός της βέλτιστης αλληλουχίας επίσκεψης τους. Επιπλέον, δημιουργήθηκε ένα μονοπάτι πλοήγησης μέσα σε κάθε δωμάτιο με στόχο την πλήρη κάλυψη των εμποδίων του χώρου, προκειμένου να πραγματοποιηθεί η απογραφή προϊόντων που είναι αποθηκευμένα στα δωμάτια με τη βοήθεια κεραιών RFID που φέρει το ρομποτικό όχημα. Το μονοπάτι αυτό δημιουργείται κάθε φορά συναρτήσει του εκάστοτε OGM και των χαρακτηριστικών των κεραιών και στοχεύει στην όσο το δυνατό μεγαλύτερη κάλυψη του χώρου και ταυτόχρονα την ελαχιστοποίηση του μήκους της διαδρομής.

Με βάση τα αποτελέσματα των πειραμάτων που αναλύθηκαν στο \autoref{chapter:experiments} και αφορούν τον εντοπισμό των δωματίων του χώρου, τον υπολογισμό της ακολουθίας επίσκεψης τους και την πλήρη κάλυψη του χάρτη, παρατηρήθηκαν τα εξής:


\begin{itemize}
    \setlength\itemsep{-0.2em}
    \item Ο διαχωρισμός του χώρου σε δωμάτια μπορεί να πραγματοποιηθεί με εξαιρετικά αποτελέσματα όταν λαμβάνονται υπόψη τα χαρακτηριστικά ολόκληρου του χάρτη.
    \item Ο RRHC αλγόριθμος βελτιώνει το συνολικό μήκος κάθε ακολουθίας δωματίων, ενώ απαιτεί αμελητέο χρόνο εκτέλεσης συγκριτικά με απλούστερους αλγορίθμους.
    \item Η δειγματοληψία σημείων με πολλαπλά βήματα επιτυγχάνει να εντοπιστούν σημεία από ολόκληρο τον χάρτη που οδηγούν σε υψηλά ποσοστά κάλυψης των εμποδίων.
    \item Όσο πιο σύνθετο είναι ένα περιβάλλον, τόσο δυσκολότερη είναι η πλήρης κάλυψη του.
    \item Η χρήση κεραιών RFID με μεγάλη ακτίνα και ευρύ FOV βελτιώνουν σημαντικά τον χρόνο πλοήγησης και το συνολικό ποσοστό κάλυψης του χώρου.
    \item Το σύστημα κάλυψης των εμποδίων είναι εύρωστο και για τις δύο στρατηγικές που αναπτύχθηκαν, καθώς τα εμπόδια σαρώνονται αρκετές φορές και με διαφορετικές γωνίες, κάτι που οδηγεί σε πολύ μεγάλη πιθανότητα σωστής απογραφής των προϊόντων.
    \item Σε περιπτώσεις που είναι σημαντική η γρήγορη κάλυψη του χώρου μπορεί να χρησιμοποιηθεί η στρατηγική ακολουθίας τοίχων. 
    \item Σε περιπτώσες που απαιτείται η αναλυτική κάλυψη του χώρου χωρίς χρονικό περιορισμό μπορεί να χρησιμοποιηθεί η ζιγκ ζαγκ στρατηγική. Αυτή αυξάνει κατά μέσο όρο τον αριθμό σαρώσεων κάθε σημείου, οι οποίες, μάλιστα, πραγματοποιούνται και με πιο ομοιόμορφη κατανομή των φάσεων σάρωσης.
\end{itemize}

% ----------------------------------------------------------------------------

\section{Προβλήματα}

Ένα βασικό πρόβλημα που διαπιστώθηκε σε ορισμένες περιπτώσεις αποτελεί η αστοχία εντοπισμού της θέσης του οχήματος κατά την πλοήγηση του στον χώρο. Συγκεκριμένα, σε σύνθετα  περιβάλλοντα που απαιτούν υψηλότερη υπολογιστική ισχύ ή σε περιπτώσεις κάλυψης μεγάλης χρονικής διάρκειας αυξάνεται η πιθανότητα το όχημα να εκτιμήσει λανθασμένα την τρέχουσα θέση του. Εαν πράγματι το όχημα αποπροσανατολιστεί, η όλη διαδικασία απογραφής θα περιέχει σημαντικά λάθη. Συνεπώς, είναι αναγκαία η μελέτη του AMCL και η περαιτέρω βελτίωση του. 

Ένα άλλο πρόβλημα που παρατηρήθηκε είναι η ταχύτητα επεξεργασίας και ανανέωσης της πληροφορίας κάλυψης. Αν και η διαδικασία αυτή δεν έχει υψηλές απαιτήσεις υπολογιστικής ισχύος, ο συνδιασμός της με τα συστήματα προσομοίωσης του χώρου και πλοήγησης οδηγεί σε περιορισμό των διαθέσιμων υπολογιστικών πόρων. Σε σύνθετα περιβάλλοντα παρατηρήθηκε η αδυναμία ανανέωσης του συστήματος κάλυψης σε πραγματικό χρόνο με αποτέλεσμα τη δημιουργία ορισμένων ασυνεχιών στο πεδίο κάλυψης. Συμπεραίνεται ότι η διαδικασία αυτή σε πραγματικούς πολύπλοκους χώρους για να λειτουργεί σωστά έχει ανάγκη από ένα σύστημα υψηλής υπολογιστικής δύναμης.

Τέλος, αποτέλεσε πρόβλημα η αναπαράσταση των διάφορων χαρτών OGM. Όπως αναφέρθηκε και στο \autoref{sec:experiments_map_annotation}, είναι σημαντικό ο χάρτης να αντιπροσωπεύει πλήρως το περιβάλλον, δίχως κενά ή λάθη. Το σύνολο των χαρτών που χρησιμοποιήθηκε στα πειράματα εντοπισμού δωματίων περιέχει χάρτες όπου ο άγνωστος χώρος αναπαριστάται ως εμπόδιο, πράγμα που οδηγεί σε λανθασμένα συμπεράσματα. Αντίστοιχα προβλήματα μπορούν να προκύψουν και από χάρτες πραγματικού SLAM, οι οποίοι περιέχουν ατέλειες και ανακρίβειες σε περιοχές που δεν έχουν χαρτογραφηθεί πλήρως. Συνεπώς, είναι σημαντικό προτού εκκινήσει η ανάλυση του εκάστοτε περιβάλλοντος να έχει εξασφαλιστεί η ύπαρξη ενός άρτιου χάρτη.