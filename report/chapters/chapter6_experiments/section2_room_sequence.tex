\section{Πειράματα Ακολουθίας Δωματίων}
\label{sec:experiments_room_sequence}

Στο δέυτερο μέρος των πειραμάτων χρησιμοποιούνται 70 χάρτες από αυτούς που χρησιμοποιήθηκαν και στο πρώτο μέρος των πειραμάτων με σκοπό τη μελέτη του μήκους της διαδρομής επίσκεψης των δωματίων κάθε χώρου. Συγκρίθηκαν 3 διαφορετικοί αλγόριθμοι, οι:

\begin{itemize}
    \setlength\itemsep{-0.2em}
    \item Επιλογή Κοντινότερου Επόμενου Δωματίου (Κοντινότερος Γείτονας)
    \item Anneal Hill Climb
    \item Random Restart Hill Climb
\end{itemize}

Ο πρώτος αποτελεί μια άπληστη μέθοδο επιλογής, όπου κάθε φορά επιλέγεται ο τρέχων καλύτερος επόμενος κόμβος στην ακολουθία, δηλαδή ο κοντινότερος. Είναι μια απλή και γρήγορη αλλά υποβέλτιστη μέθοδος, εφαρομογή ουσιαστικά του αλγορίθμου επιλογής κοντινότερου γείτονα. Οι δύο άλλοι αναπτύχθηκαν στην \autoref{subsection:room_sequence_implementation} και η σύγκριση τους με την άπληστη μέθοδο καταδεικνύει την αποτελεσματικότητα τους. Επίσης, επισημαίνεται ότι οι δύο HC αλγόριθμοι εκτελέστηκαν για πλήθος επαναλήψεων ίσο με 500 φορές επί το πλήθος των πορτών του χάρτη. Σημαντική παρατήρηση είναι ότι ο χρόνος εκτέλεσης και των τριών αλγορίθμων είναι παρόμοιος και αμελητέος.


Ορισμένα τελικά αποτελέσματα μήκους της συνολικής ακολουθίας επίσκεψης των πορτών σε τιμές brushfire, δηλαδή σε μήκος πάνω στο εκάστοτε OGM είναι:


\begin{longtable}{| p{.20\textwidth} | p{.20\textwidth} | p{.20\textwidth} | p{.20\textwidth} |}

    \caption{Επιλεγμένα Αποτελέσματα Μήκους Ακολουθίας Δωματίων}
    \label{tab:room_sequence_results}
    \hline
    \rowcolor[gray]{0.8}
    Κοντινότερος Γείτονας & Anneal HC & RRHC & Ποσοστό Βελτίωσης με RRHC \\ \hline
    $2809$ & $5201$ & $2809$ & $0$\%\\ \hline
    $3338$ & $6870$ & $2979$ & $10.6$\%\\ \hline
    $7928$ & $11990$ & $7698$ & $2.9$\%\\ \hline
    $2414$ & $2766$ & $2318$ & $3.9$\%\\ \hline
    $2769$ & $4087$ & $2519$ & $9.0$\%\\ \hline
    $2834$ & $3766$ & $2834$ & $0$\%\\ \hline
    $11466$ & $13762$ & $10331$ & $9.8$\%\\ \hline
    $996$ & $996$ & $849$ & $14.7$\%\\ \hline
    $558$ & $558$ & $558$ & $0$\%\\ \hline
    $1149$ & $1149$ & $849$ & $26.1$\%\\ 
    \hline
\end{longtable}



Από τα πειράματα προέκυψαν τα παρακάτω ευρήματα. Αρχικά, ο αλγόριθμος anneal HC τις περισσότερες φορές δίνει αποτέλεσμα σημαντικά μεγαλύτερο από το απλό εξαιτίας της στοχαστικότητας του, κάτι που τελικά τον θέτει μη ικανοποιητικό. Μόλις σε 12 χάρτες, στις πιο απλές περιπτώσεις, οι τιμές του anneal HC ήταν ίσες με τις τιμές του κοντινότερου γείτονα. 

Επιπλέον, ο RRHC δίνει σταθερά ίδια ή καλύτερα αποτελέσματα από τον αλγόριθμο κοντινότερου γείτονα, επομένως, αποτελεί μια σημαντική βελτιωμένη διαδικασία. Σε απλές περιπτώσεις ή περιπτώσεις με μικρό αριθμό δωματίων τα αποτελέσματα των δύο αλγορίθμων είναι συνήθως ίσα, ωστόσο όσο αυξάνεται το πλήθος των δωματίων και μια βέλτιστη λύση γίνεται πιο σύνθετη, τόσο σημαντιότερη είναι η διαφορά των αποτελεσμάτων τον δύο αλγορίθμων. Συγκεκριμένα, σε 31 χάρτες παρατηρήθηκε βελτίωση στο μήκος της διαδρομής, ενώ στους υπόλοιπους 39 τα αποτελέσματα ήταν ίσα. Κατά μέσο όρο το ποσοστό βελτίωσης του μήκους της διαδρομής είναι $7.597$\% στους χάρτες που πράγματι παρουσιάζεται βελτίωση και $3.256$\% συνολικά, ενώ τα τρία μεγαλύτερα ποσοστά είναι $26.1$\%, $15.9$\% και $14.7$\%.



Τα συνολικά αποτελέσματα είναι:
\begin{table}[H]
  \begin{center}
    \caption{Μετρήσεις Μήκους Ακολουθίας Δωματίων}
    \label{tab:room_detection}
    \begin{tabular}{ |>{\columncolor[gray]{0.8}} c | c | }
      \hline
      Χάρτες που παρουσίασαν βελίωση & $31\ /\ 70$ \\ \hline
      Μέση Τιμή βελτίωσης & $3.256\%$ \\ \hline
      Τυπική απόκλιση βελτίωσης & $5.166\%$ \\
      \hline
    \end{tabular}
  \end{center}
\end{table}

Συμπερασματικά, ο anneal HC δεν προσφέρει κάποια βελτίωση, ενώ αντίθετα ο RRHC οδηγεί σε ικανοποιητικά αποτελέσματα και από τη στιγμή που δεν επιφέρει χρονική καθυστέρηση συγκριτικά με απλούστερες μεθόδους, όπως του κοντινότερου γείτονα, μπορεί να χρησιμοποιηθεί για τον υπολογισμό της βέλτιστης ακολουθίας δωματίων κάθε χάρτη.
