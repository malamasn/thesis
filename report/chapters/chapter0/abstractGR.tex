\begin{center}
  \centering

  \vspace{0.5cm}
  \centering
  \textbf{\Large{Περίληψη}}
  \phantomsection
  \addcontentsline{toc}{section}{Περίληψη}

  \vspace{1cm}

\end{center}

Ο κλάδος της ρομποτικής έχει εμφανίσει ραγδαία εξέλιξη τα τελευταία χρόνια. Η χρήση της αν και στην αρχή περιοριζόταν κυρίως σε στρατιωτικές εφαρμογές, σήμερα έχει γενικευτεί σε μια πληθώρα εφαρμογών τόσο επαγγελματικού όσο και καθημερινού περιεχομένου. Η σχέση των ανθρώπων με την ρομποτική συνεχώς αλλάζει και εξελίσσεται, σε μια προσπάθεια αυτή να βοηθήσει στην αποτελεσματικότερη αντιμετώπιση των διάφορων ανθρωπίνων προβλημάτων. 

Ένας ολόκληρος τομέας των εφαρμογών της ρομποτικής αφορά την αυτόνομη πλοήγηση ρομποτικών οχημάτων σε γνωστούς ή και άγνωστους χώρους. Τέτοιες περιπτώσεις είναι τα αυτοκινούμενα αυτοκίνητα, ο αυτόματος καθαρισμός ενός χώρου από μια ρομποτική σκούπα, η αυτόνομη απογραφή προϊόντων σε μια αποθήκη, η χαρτογράφηση άγνωστων περιοχών κ.α.

Η παρούσα διπλωματική εργασία έχει ως στόχο την μελέτη και επίλυση του προβλήματος της αυτοματοποίησης της διαδικασίας απογραφής προϊόντων σε οποιονδήποτε γνωστό δισδιάστατο χώρο αποτελεσματικά. Το πρόβλημα αυτό περιλαμβάνει τρία υποπροβλήματα: α) τον διαχωρισμό του γνωστού χώρου σε υποχώρους, β) τον υπολογισμό της αλληλουχίας προσέγγισης των διάφορων υποχώρων, και γ) την εύρεση του μονοπατιού για την πλήρη κάλυψη του χώρου. Η κάλυψη αυτή, μάλιστα, πραγματοποιείται από αισθητήρες τα χαρακτηριστικά των οποίων δεν είναι πρότερα γνωστά.

Τα προβλήματα αυτά αντιμετωπίζονται κάνοντας χρήση του γνωστού δισδιάστατου χάρτη του χώρου. Αρχικά, υλοποιείται η τοπολογική ανάλυση του χώρου προκειμένου να εντοπιστούν τα διάφορα δωμάτια του περιβάλλοντος. Έπειτα, υπολογίζεται η βέλτιστη σειρά επίσκεψης αυτών των δωματίων. Τέλος, υπολογίζεται το βέλτιστο μονοπάτι κίνησης σε κάθε υποχώρο μέσω των οποίων το όχημα καλύπτει πλήρως το χάρτη. Τα κριτήρια αξιολόγησης της μελέτης είναι η καλύτερη δυνατή κάλυψη του χώρου και η ταχύτητα εκτέλεσης τόσο της διαδικασίας πλοήγησης όσο και των υπολογισμών. Επιπλέον, στην διπλωματική αυτή εργασία εξετάζονται δύο διαφορετικές στραγητικές κάλυψης του χώρου που αφορούν τον τρόπο προσέγγισης του ρομποτικού οχήματος στα πιθανά σημεία εύρεσης προϊόντων.

Επιπρόσθετα, πραγματοποιήθηκαν πειράματα στα τρία τμήματα της διαδικασίας σε περιβάλλον προσομοίωσης τα αποτελέσματα των οποίων παρουσιάζονται. Χρησιμοποιήθηκαν χώροι με διαφορετική μορφολογία και αισθητήρες με διαφορετικά χαρακτηριστικά, με στόχο την ολοκληρωμένη αξιολόγηση της μελέτης.


\begin{flushright}
  \vspace{1.5cm}
  Νικόλαος Μάλαμας
  \\
  malamasn@ece.auth.gr
  \\
  Τμήμα Ηλεκτρολόγων Μηχανικών και Μηχανικών Υπολογιστών,
  \\
  Αριστοτέλειο Πανεπιστήμιο Θεσσαλονίκης, Ελλάδα
  \\
  Σεπτέμβριος 2019
\end{flushright}

