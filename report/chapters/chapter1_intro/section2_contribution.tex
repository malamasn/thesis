\section{Σκοπός - Συνεισφορά της Διπλωματικής Εργασίας}
\label{section:contribution}

Στόχος αυτής της διπλωματικής εργασίας είναι η παρουσίαση ενός ολοκληρωμένου συστήματος ανάλυσης του χάρτη του χώρου, υπολογισμού του βέλτιστου μονοπατιού πλοήγησης συναρτήσει των αισθητήρων που φέρει το ρομπότ και πλήρους κάλυψης του χώρου χρησιμοποιώντας ένα μη επανδρωμένο επίγειο όχημα. 

Η ανάλυση του χώρου πραγματοποιείται βρίσκοντας την τοπολογία του χώρου, κάνοντας χρήση του χάρτη σε μορφή OGM. Έτσι, ο χώρος διαχωρίζεται σε επιμέρους τμήματα, τα οποία και μελετούνται στη συνέχεια ως ξεχωριστές οντότητες.

Η εύρεση του βέλτιστου μονοπατιού περιλαμβάνει τον υπολογισμό της βέλτιστης ακολουθίας κάλυψης των δωματίων. Αυτό επιτυγχάνεται χρησιμοποιώντας μια παραλλαγή του αλγορίθμου αναρρίχησης λόφων (Hill-Climbing). 

Στη συνέχεια για κάθε δωμάτιο υπολογίζονται θέσεις στον χώρο απ' όπου το όχημα θα έχει πρόσβαση στα προϊόντα με ομοιόμορφη δειγματοληψία πολλαπλών βημάτων και υπολογίζεται η βέλτιστη ακολουθία τους. Για την εύρεση της βέλτιστης ακολουθίας χρησιμοποιούνται παραλλαγές του αλγορίθμου αναρρίχησης λόφων (Hill-Climbing) σε συνδιασμό με τον αλγόριθμο εύρεσης πλησιέστερου γείτονα (Nearest Neighbor). Μετά, υπολογίζονται οι καλύτεροι δυνατοί προσανατολισμοί του οχήματος σε κάθε θέση με στόχο την καταγραφή όσων περισσότερων προϊόντων γίνεται και ταυτόχρονα την ελαχιστοποίηση του μήκους του συνολικού μονοπατιού.
 
Έπειτα, πραγματοποιείται μια μείωση του πλήθους των σημείων τα οποία έχει να επισκεφθεί το όχημα, αφαιρώντας τα περιττά, προκειμένου να μειωθεί ο συνολικός χρόνος πλοήγησης, χωρίς όμως να επηρεάζει το συνολικό ποσοστό κάλυψης του χώρου.
 
Στην εργασία αυτή χρησιμοποιούνται δύο διαφορετικές κύριες στρατηγικές προσέγγισης των σημείων αυτών του χώρου. Η πρώτη αποτελεί μια προσέγγιση της στρατηγικής ακολουθίας των τοίχων του χώρου (wall following), καθώς σε αυτούς βρίσκονται τα προϊόντα. Στην δεύτερη δημιουργούνται κινήσεις ζιγκ-ζαγκ προκειμένου να σαρώνουν οι αισθητήρες τον χώρο περισσότερες φορές και από διαφορετικές γωνίες.
 
Τέλος, ο κώδικας που αναπτύχθηκε εκτελέσθηκε σε περιβάλλον προσομοίωσης τόσο κατά τη διάρκεια σχεδιασμού των αλγορίθμων για την επιμέρους βελτίωση, όσο και συνολικά στο πέρας της μελέτης.