\section{Διάρθρωση της Αναφοράς}
\label{section:layout}

Η διάρθρωση της παρούσας διπλωματικής εργασίας είναι η εξής:

\begin{itemize}
  \item{\textbf{Κεφάλαιο \ref{chapter:sota}:}
      Γίνεται ανασκόπηση της ερευνητικής περιοχής που αφορά την επιλογή κατάλληλων αισθητήρων, την τοπολογική ανάλυση ενός χώρου και την πλήρη κάλυψη του.
    }
  \item{\textbf{Κεφάλαιο \ref{chapter:theory}:} Περιγράφονται βασικά θεωρητικά στοιχεία
      στα οποία βασίστηκαν οι υλοποιήσεις.
    }
  \item{\textbf{Κεφάλαιο \ref{chapter:tools}:} Παρουσιάζονται τα εργαλεία που χρησιμοποιήθηκαν στις
      υλοποιήσεις.
    }
  \item{\textbf{Κεφάλαιο \ref{chapter:implementations}:} Αναλύονται οι υλοποιήσεις που αναπτύχθηκαν.
    }
  \item{\textbf{Κεφάλαιο \ref{chapter:experiments}:} Παρουσιάζεται αναλυτικά η μεθοδολογία των πειραμάτων και τα αποτελέσματα.
    }
  \item{\textbf{Κεφάλαιο \ref{chapter:conclusions}:} Παρουσιάζονται τα τελικά συμπεράσματα.
    }
  \item{\textbf{Κεφάλαιο \ref{chapter:future_work}:} Αναφέρονται τα
      προβλήματα που προέκυψαν και προτείνονται θέματα για μελλοντική
      μελέτη, αλλαγές και επεκτάσεις.
    }
\end{itemize}

