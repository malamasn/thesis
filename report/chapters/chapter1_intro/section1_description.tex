\section{Περιγραφή του Προβλήματος}
\label{section:problem_description}

Η χρήση των \emph{Μη Επανδρομένων Επίγειων Οχημάτων} αυξάνεται συνεχώς με όλο και περισσότερες εφαρμογές τα τελευταία χρόνια. Κάποιες από αυτές είναι οι εξής:
\begin{itemize}
    \setlength\itemsep{-0.2em}
    \item Εύρεση του χάρτη ενός δυσπρόσιτου χώρου (SLAM), όπως ένα ορυχείο.
    \item Αναζήτηση ανθρώπων σε χώρους που έχουν υποστεί φυσικές καταστροφές.
    \item Αυτόνομη μεταφορά και παράδοση δεμάτων.
    \item Αυτοματοποιημένη απογραφή προϊόντων σε αποθηκευτικούς χώρους.
    \item Αποτελεσματικός καθαρισμός του πατώματος ενός σπιτιού.
    \item Χρήση τους ως μέσα μεταφοράς χωρίς οδηγό.
    \item Χρήση τους ως ξεναγός ή βοηθός σε χώρους με πολύ κόσμο, όπως μουσεία, αεροδρόμια κ.α.
\end{itemize}
Αυτές οι εφαρμογές επιτυγχάνονται, διότι τα οχήματα αυτά φέρουν κατάλληλους αισθητές που είναι προσαρμοσμένοι στην κάθε περίπτωση, αλλά και γιατί έχουν αναπτυχθεί πολλοί αποτελεσματικοί αλγόριθμοι πλοήγησης. Έτσι, τα ρομπότ είναι σε θέση να δημιουργήσουν μια αντίληψη για τον περιβάλλοντα χώρο και να μάθουν να κινούνται μέσα σ' αύτον εκπληρώνοντας τον εκάστοτε στόχο τους.

Στην διπλωματική αυτή εργασία πραγματοποιείται μια αναλυτική μελέτη της αυτόνομης απογραφής των προϊόντων σε έναν οποιονδήποτε γνωστό χώρο. Η απογραφή προϊόντων είναι μια διαδικασία η οποία μπορεί να εκτελεστεί από ρομποτικούς πράκτορες, αντί για ανθρώπους. Αποτελεί μια καθορισμένη διαδικασία, η οποία δεν απαιτεί ανθρώπινες ικανότητες που δεν μπορούν να υλοποιηθούν από ένα ρομπότ, όπως είναι η λήψη πολύπλοκων αποφάσεων και η εκτέλεση σύνθετων κινήσεων. Έτσι, το ανθρώπινο δυναμικό θα έχει την δυνατότητα να ασχοληθεί με πιο πολύπλοκες και απαιτητικές διεργασίες την ώρα που ένα ρομποτικό όχημα μπορεί να φέρει εις πέρας την διαδικασία αυτή και να καταγράψει την θέση όλων των αποθηκευμένων προϊόντων με ακρίβεια μερικών εκατοστών. 

Το πρόβλημα που μελετάται διακρίνεται σε τρία υποπροβλήματα: α) τον διαχωρισμό του χώρου σε υποχώρους (map annotation/decomposition), β) τον υπολογισμό της βέλτιστης ακολουθίας επίσκεψης τους και, γ) την εύρεση του βέλτιστου μονοπατιού σε κάθε υποχώρο (path planning). Ο διαχωρισμός του χώρου πραγματοποιείται προκειμένου να απλοποιηθεί το πρόβλημα υπολογισμού του βέλτιστου μονοπατιού σε πολλά μικρότερα, καθώς η προσπάθεια εύρεσης ενός συνολικού μονοπατιού θα ήταν υπολογιστικά απαγορευτική. Επιπλέον, σε κάθε υποχώρο υπολογίζεται μια αλληλουχία σημείων τα οποία πρέπει να προσπελάσει το όχημα, ώστε να καλύψει πλήρως τον χώρο. Τα κριτήρια αξιολόγησης των διάφορων μεθόδων που μελετώνται είναι ο συνολικός χρόνος πλοήγησης και το τελικό ποσοστό κάλυψης του χώρου στο οποίο μπορούν να συναντηθούν προϊόντα.

Η προσέγγιση του προβλήματος είναι ανεξάρτητη τόσο του χώρου όσο και των αισθητήρων κάλυψης που φέρει το όχημα. Ο χώρος τον οποίο πρέπει να καλύψει το όχημα είναι γνωστός και βρίσκεται σε μορφή \emph{χάρτη - πλέγμα πιθανοτικής κάλυψης} (Occupancy Grid Map - OGM) \ref{section:ogm}. Επίσης, οι αισθητήρες κάλυψης είναι RFID κεραίες και τα χαρακτηριστικά τους δίνονται από τον χρήστη. Όλοι οι υπολογισμοί των μονοπατιών είναι παραμετρικοί αυτών. Η ανεξαρτησία αυτή στοχεύει σε μια γενικευμένη λύση, δίχως εξαρτήσεις σε στοιχεία που εκ των πραγμάτων δεν είναι πρότερα γνωστά, στην περίπτωση της πραγματικής εφαρμογής της μεθόδου αυτής.

