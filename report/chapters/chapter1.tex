\chapter{Εισαγωγή}
\label{chapter:intro}

Ο όρος \emph{Ρομποτική} αναφέρεται στον κλάδο της επιστήμης που μελετά τις μηχανές εκείνες που μπορούν να αντικαταστήσουν τον άνθρωπο στην εκτέλεση μιας εργασίας, η οποία συνδυάζει τη φυσική δραστηριότητα με τη διαδικασία λήψης αποφάσεων. 
Οι μηχανές αυτές ονομάζονται ρομπότ και ένας ορισμός τους είναι ο επόμενος.
\begin{displayquote} Ρομπότ είναι ένας πράκτορας εφοδιασμένος με ένα πλήθος αισθητήρων και τελεστών, για να ερμηνεύει το περιβάλλον του και να κινείται μέσα σε αυτό, ενώ μπορεί να προγραμματιστεί για να παρουσιάσει νοημοσύνη. 
\end{displayquote}
Η ρομποτική, λοιπόν, έχει ώς στόχο τον προγραμματισμό αυτό των μηχανών προκειμένου να παρουσιάσουν αυτόνομα νοημοσύνη και να εκτελέσουν κάποια διεργασία χωρίς την ανθρώπινη εποπτεία.

Οι πρώτες προσπάθειες του ανθρώπου να κατασκευάσουν αυτόνομες μηχανές παρατηρούνται στην αρχαία Ελλάδα με πιο γνωστή την κατασκευή του μηχανισμού των Αντικυθήρων. Η ρομποτική, στη μορφή που έχει σήμερα, άρχισε να αναπτύσσεται στα μέσα του 20ού αιώνα με στόχο την αυτοματοποίηση της παραγωγικής διαδικασίας στις βιομηχανίες μαζικής παραγωγής. Όμως, δεν περιορίζεται πια μόνο σ' αυτόν τον τομέα. Πετυχημένα ρομποτικά συστήματα συναντά κανείς στην ιατρική, στις αγροτικές καλλιέργιες, στην εξερεύνηση του διαστήματος, στην διάσωση ανθρώπων από καταστροφές. Μάλιστα, γίνονται καθημερινά προσπάθειες χρήσης ρομποτικών συστημάτων σε όλο και περισσότερες και πιο πολύπλοκες διεργασίες σε μια προσπάθεια βελτίωσης της αποτελεσματικότητας τους. 

Η καθολική χρήση της ρομποτικής εγείρει ορισμένα θέματα ηθικής. Ο Isaac Asimov διατύπωσε το 1942 τους τρεις νόμους της Ρομποτικής \cite{mccauley2007} οι οποίοι έθεσαν για πρώτη φορά ένα ηθικό πλαίσιο χρήσης της και βρίσκονται μέχρι και σήμερα σε συνεχή μελέτη και αναθεώρηση. Οι νόμοι είναι οι εξής:
\begin{displayquote}
    \begin{itemize}
        \item{Ένα ρομπότ δεν πρέπει να βλάπτει ένα ανθρώπινο ον, ή να επιτρέψει σε ένα ανθρώπινο ον να τραυματιστεί μέσω αδράνειάς του.}
        \item{Ένα ρομπότ πρέπει να υπακούει σε εντολές που του δίνονται απο ανθρώπινα όντα, εκτός αν αυτές οι εντολές παραβαίνουν τον πρώτο νόμο.}
        \item{Ένα ρομπότ πρέπει να προστατεύεται, εκτός απο τις περιπτώσεις που παραβιάζεται ο πρώτος ή ο δεύτερος νόμος.}
    \end{itemize}
\end{displayquote}


\section{Περιγραφή του Προβλήματος}
\label{section:problem_description}

Η χρήση των \emph{Μη Επανδρομένων Επίγειων Οχημάτων} αυξάνεται συνεχώς με όλο και περισσότερες εφαρμογές τα τελευταία χρόνια. Κάποιες από αυτές είναι οι εξής:
\begin{itemize}
    \setlength\itemsep{-0.2em}
    \item Εύρεση του χάρτη ενός δυσπρόσιτου χώρου (SLAM), όπως ένα ορυχείο.
    \item Αναζήτηση ανθρώπων σε χώρους που έχουν υποστεί φυσικές καταστροφές.
    \item Αυτόνομη μεταφορά και παράδοση δεμάτων.
    \item Αυτοματοποιημένη απογραφή προϊόντων σε αποθηκευτικούς χώρους.
    \item Αποτελεσματικός καθαρισμός του πατώματος ενός σπιτιού.
    \item Χρήση τους ως μέσα μεταφοράς χωρίς οδηγό.
    \item Χρήση τους ως ξεναγός ή βοηθός σε χώρους με πολύ κόσμο, όπως μουσεία, αεροδρόμια κ.α.
\end{itemize}
Αυτές οι εφαρμογές επιτυγχάνονται, διότι τα οχήματα αυτά φέρουν κατάλληλους αισθητές που είναι προσαρμοσμένοι στην κάθε περίπτωση, αλλά και γιατί έχουν αναπτυχθεί πολλοί αποτελεσματικοί αλγόριθμοι πλοήγησης. Έτσι, τα ρομπότ είναι σε θέση να δημιουργήσουν μια αντίληψη για τον περιβάλλοντα χώρο και να μάθουν να κινούνται μέσα σ' αύτον εκπληρώνοντας τον εκάστοτε στόχο τους.

Στην διπλωματική αυτή εργασία πραγματοποιείται μια αναλυτική μελέτη της αυτόνομης απογραφής των προϊόντων σε έναν οποιονδήποτε γνωστό χώρο. Η απογραφή προϊόντων είναι μια διαδικασία η οποία μπορεί να εκτελεστεί από ρομποτικούς πράκτορες, αντί για ανθρώπους. Αποτελεί μια καθορισμένη διαδικασία, η οποία δεν απαιτεί ανθρώπινες ικανότητες που δεν μπορούν να υλοποιηθούν από ένα ρομπότ, όπως είναι η λήψη πολύπλοκων αποφάσεων και η εκτέλεση σύνθετων κινήσεων. Έτσι, το ανθρώπινο δυναμικό θα έχει την δυνατότητα να ασχοληθεί με πιο πολύπλοκες και απαιτητικές διεργασίες την ώρα που ένα ρομποτικό όχημα μπορεί να φέρει εις πέρας την διαδικασία αυτή και να καταγράψει την θέση όλων των αποθηκευμένων προϊόντων με ακρίβεια μερικών εκατοστών. 

Το πρόβλημα που μελετάται διακρίνεται σε τρία υποπροβλήματα: α) τον διαχωρισμό του χώρου σε υποχώρους (map annotation/decomposition), β) τον υπολογισμό της βέλτιστης ακολουθίας επίσκεψης τους και, γ) την εύρεση του βέλτιστου μονοπατιού σε κάθε υποχώρο (path planning). Ο διαχωρισμός του χώρου πραγματοποιείται προκειμένου να απλοποιηθεί το πρόβλημα υπολογισμού του βέλτιστου μονοπατιού σε πολλά μικρότερα, καθώς η προσπάθεια εύρεσης ενός συνολικού μονοπατιού θα ήταν υπολογιστικά απαγορευτική. Επιπλέον, σε κάθε υποχώρο υπολογίζεται μια αλληλουχία σημείων τα οποία πρέπει να προσπελάσει το όχημα, ώστε να καλύψει πλήρως τον χώρο. Τα κριτήρια αξιολόγησης των διάφορων μεθόδων που μελετώνται είναι ο συνολικός χρόνος πλοήγησης και το τελικό ποσοστό κάλυψης του χώρου στο οποίο μπορούν να συναντηθούν προϊόντα.

Η προσέγγιση του προβλήματος είναι ανεξάρτητη τόσο του χώρου όσο και των αισθητήρων κάλυψης που φέρει το όχημα. Ο χώρος τον οποίο πρέπει να καλύψει το όχημα είναι γνωστός και βρίσκεται σε μορφή \emph{χάρτη - πλέγμα πιθανοτικής κάλυψης} (Occupancy Grid Map - OGM) \ref{section:ogm}. Επίσης, οι αισθητήρες κάλυψης είναι RFID κεραίες και τα χαρακτηριστικά τους δίνονται από τον χρήστη. Όλοι οι υπολογισμοί των μονοπατιών είναι παραμετρικοί αυτών. Η ανεξαρτησία αυτή στοχεύει σε μια γενικευμένη λύση, δίχως εξαρτήσεις σε στοιχεία που εκ των πραγμάτων δεν είναι πρότερα γνωστά, στην περίπτωση της πραγματικής εφαρμογής της μεθόδου αυτής.


\section{Σκοπός - Συνεισφορά της Διπλωματικής Εργασίας}
\label{section:contribution}

Στόχος αυτής της διπλωματικής εργασίας είναι η παρουσίαση ενός ολοκληρωμένου συστήματος ανάλυσης του χάρτη του χώρου, υπολογισμού του βέλτιστου μονοπατιού πλοήγησης συναρτήσει των αισθητήρων που φέρει το ρομπότ και πλήρους κάλυψης του χώρου χρησιμοποιώντας ένα μη επανδρωμένο επίγειο όχημα. 

Η ανάλυση του χώρου πραγματοποιείται βρίσκοντας την τοπολογία του χώρου, κάνοντας χρήση του χάρτη σε μορφή OGM. Έτσι, ο χώρος διαχωρίζεται σε επιμέρους τμήματα, τα οποία και μελετούνται στη συνέχεια ως ξεχωριστές οντότητες.

Η εύρεση του βέλτιστου μονοπατιού περιλαμβάνει τον υπολογισμό της βέλτιστης ακολουθίας κάλυψης των δωματίων. Αυτό επιτυγχάνεται χρησιμοποιώντας μια παραλλαγή του αλγορίθμου αναρρίχησης λόφων (Hill-Climbing). 

Στη συνέχεια για κάθε δωμάτιο υπολογίζονται θέσεις στον χώρο απ' όπου το όχημα θα έχει πρόσβαση στα προϊόντα με ομοιόμορφη δειγματοληψία πολλαπλών βημάτων και υπολογίζεται η βέλτιστη ακολουθία τους. Για την εύρεση της βέλτιστης ακολουθίας χρησιμοποιούνται παραλλαγές του αλγορίθμου αναρρίχησης λόφων (Hill-Climbing) σε συνδιασμό με τον αλγόριθμο εύρεσης πλησιέστερου γείτονα (Nearest Neighbor). Μετά, υπολογίζονται οι καλύτεροι δυνατοί προσανατολισμοί του οχήματος σε κάθε θέση με στόχο την καταγραφή όσων περισσότερων προϊόντων γίνεται και ταυτόχρονα την ελαχιστοποίηση του μήκους του συνολικού μονοπατιού.
 
Έπειτα, πραγματοποιείται μια μείωση του πλήθους των σημείων τα οποία έχει να επισκεφθεί το όχημα, αφαιρώντας τα περιττά, προκειμένου να μειωθεί ο συνολικός χρόνος πλοήγησης, χωρίς όμως να επηρεάζει το συνολικό ποσοστό κάλυψης του χώρου.
 
Στην εργασία αυτή χρησιμοποιούνται δύο διαφορετικές κύριες στρατηγικές προσέγγισης των σημείων αυτών του χώρου. Η πρώτη αποτελεί μια προσέγγιση της στρατηγικής ακολουθίας των τοίχων του χώρου (wall following), καθώς σε αυτούς βρίσκονται τα προϊόντα. Στην δεύτερη δημιουργούνται κινήσεις ζιγκ-ζαγκ προκειμένου να σαρώνουν οι αισθητήρες τον χώρο περισσότερες φορές και από διαφορετικές γωνίες.
 
Τέλος, ο κώδικας που αναπτύχθηκε εκτελέσθηκε σε περιβάλλον προσομοίωσης τόσο κατά τη διάρκεια σχεδιασμού των αλγορίθμων για την επιμέρους βελτίωση, όσο και συνολικά στο πέρας της μελέτης.
\section{Διάρθρωση της Αναφοράς}
\label{section:layout}

Η διάρθρωση της παρούσας διπλωματικής εργασίας είναι η εξής:

\begin{itemize}
  \item{\textbf{Κεφάλαιο \ref{chapter:sota}:}
      Γίνεται ανασκόπηση της ερευνητικής περιοχής που αφορά την επιλογή κατάλληλων αισθητήρων, την τοπολογική ανάλυση ενός χώρου και την πλήρη κάλυψη του.
    }
  \item{\textbf{Κεφάλαιο \ref{chapter:theory}:} Περιγράφονται βασικά θεωρητικά στοιχεία
      στα οποία βασίστηκαν οι υλοποιήσεις.
    }
  \item{\textbf{Κεφάλαιο \ref{chapter:tools}:} Παρουσιάζονται τα εργαλεία που χρησιμοποιήθηκαν στις
      υλοποιήσεις.
    }
  \item{\textbf{Κεφάλαιο \ref{chapter:implementations}:} Αναλύονται οι υλοποιήσεις που αναπτύχθηκαν.
    }
  \item{\textbf{Κεφάλαιο \ref{chapter:experiments}:} Παρουσιάζεται αναλυτικά η μεθοδολογία των πειραμάτων και τα αποτελέσματα.
    }
  \item{\textbf{Κεφάλαιο \ref{chapter:conclusions}:} Παρουσιάζονται τα τελικά συμπεράσματα.
    }
  \item{\textbf{Κεφάλαιο \ref{chapter:future_work}:} Αναφέρονται τα
      προβλήματα που προέκυψαν και προτείνονται θέματα για μελλοντική
      μελέτη, αλλαγές και επεκτάσεις.
    }
\end{itemize}


