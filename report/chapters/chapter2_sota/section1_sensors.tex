\section{Επιλογή Αισθητήρων}
\label{section:sensor_selection}


Το πρώτο και σημαντικότερο πρόβλημα είναι η επιλογή των σωστών αισθητήρων για την εφαρμογή αυτη. Δίχως σωστούς αισθητήρες η συνολική διαδικασία της κάλυψης του χώρου θα περιέχει σφάλματα, ανεξάρτητα από την υπόλοιπη διαδικασία. Συνεπώς, η επιλογή κατάλληλων αισθητήρων χρίζει προσοχής.

Μια μέθοδος που έχει αναπτυχθεί στο παρελθόν είναι η χρήση ετικετών \emph{barcode}. Η τεχνολογία αυτή μελετήθηκε από τον Emery \cite{barcodes} για τη βελτίωση του χρόνο και του κόστους της διαδικασίας της καταγραφής προϊόντων στις αποθήκες. Η ανάγνωση τους μπορεί να πραγματοποιηθεί είτε με την επεξεργασία της εικόνας μιας κάμερας είτε με τη χρήση laser ανάγνωσης barcodes (barcode scanner). Και οι δύο αυτοί αισθητήρες, όμως, βασίζονται στην οπτική επαφή τους με το εκάστοτε προϊόν, πράγμα που συχνά είναι αδύνατο.

Η χρήση \emph{RFID κεραιών} για την καταγραφή αντικειμένων σε εσωτερικούς χώρους προτάθηκε από τον Tesoriero κ.α. \cite{tesoriero2009} μαζί με ένα σύστημα εντοπισμού θέσης σε κλειστούς χώρους, σε μια προσπάθεια αντικατάστασης του GPS το οποίο δεν λειτουργεί σωστά σ' αυτές τις περιπτώσεις. Το σύστημα που προτείνουν συσχετίζει τον αριθμό της ετικέτας RFID που διαβάζει η κεραία με την τρέχουσα θέση του ρομπότ. Έτσι, μπορεί να αποθηκευτεί το σύνολο των αντικειμένων που συνάντησε κατά την περιήγηση του το όχημα και να πραγματοποιηθεί μια ολοκληρωμένη απογραφή. Η μέθοδος αυτή είναι αρκετά ακριβής και οικονομικά φιλική για κάθε αποθήκη, ενώ δεν απαιτεί την οπτική επαφή μεταξύ κεραιών και προϊόντων.
