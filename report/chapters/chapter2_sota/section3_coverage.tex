\section{Πλήρης Κάλυψη Χώρου}
\label{section:full_coverage}

Η πλήρης κάλυψη ενός χάρτη OGM, αλλά και ενός χώρου γενικότερα, συνεπάγεται την κάλυψη κάθε σημείου του χώρου του χάρτη αυτού. Όπως διατύπωσαν οι Galceran και Carreras \cite{Galceran2013ASO} υπάρχουν έξι προϋποθέσεις για να καλυφθεί βέλτιστα ένας χώρος.
\begin{itemize}
    \setlength\itemsep{-0.2em}
    \item Το ρομπότ πρέπει να περάσει απ' όλα τα σημεία του χώρου, καλύπτοντας τα πλήρως.
    \item Το ρομπότ πρέπει να καλύψει την περιοχή δίχως την ύπαρξη αλληλοεπικαλυπτόμενων περιοχών.
    \item Απαιτούνται συνεχείς και ακολουθιακές διεργασίες, δίχως επαναλήψεις μονοπατιών.
    \item Το ρομπότ πρέπει να αποφεύγει τα εμπόδια του περιβάλλοντος.
    \item Πρέπει να χρησιμοποιούνται όσο το δυνατόν πιο απλές τροχιές κίνησης, όπως ευθείες.
    \item Προτιμάται ένα βέλτιστο μονοπάτι, εαν αυτό υπάρχει.
\end{itemize}

Όμως, το πρόβλημα εύρεσης της βέλτιστης διαδρομής ενός συνόλου σημείων αποτελεί \emph{το πρόβλημα του περιπλανώμενου πωλητή} (Traveling Salesman Problem - TSP) και είναι NP-hard πρόβλημα. Γι' αυτό έχουν αναπτυχθεί απόλυτες μέθοδοι που διαχωρίζουν τον χώρο σε μικρότερα χωρία και επιλύουν βέλτιστα το πρόβλημα της κάλυψης άνα χωρίο, αλλά και ευριστικές μέθοδοι που προσπαθούν να απλοποιήσουν σημαντικά το πρόβλημα και πολλές φορές δεν φτάνουν σε μια βέλτιστη λύση.

Μια από τις πρώτες προσεγγίσεις που δημοσιεύτηκαν είναι των Choset και Pignon \cite{Choset1998}. Αυτοί πρότειναν την μέθοδο της αποσύνθεσης του χώρου με την κινήση boustrophedon, την κίνηση δηλαδή που κάνει ένα βόδι σε ένα χωράφι. Οι περιοχές οι οποίες περιλαμβάνουν εμπόδια δεν επιτρέπουν την ύπαρξη μιας συνεχόμενης διαδρομής. Γι' αυτό, ο χώρος χωρίζεται σε επιμέρους τμήματα και στο εσωτερικό τους πραγματοποιείται η συγκεκριμένη κίνηση. Έτσι, καλύπτονται τα σημεία του χώρου με απλές κινήσεις του ρομποτικού οχήματος προς τα μπροστά και προς τα πίσω. Μόλις καλυφθεί κάθε σημείο, αυτομάτως έχει καλυφθεί ολόκληρος ο χώρος.

Στο \cite{brown2016} χρησιμοποιείται μια παραλλαγή της boustrophedon μεθόδου. Εδώ σε κάθε χώρο σχηματίζεται μια σπειροειδής διαδρομή που ξεκινάει από το κέντρο του δωματίου και τερματίζει στην πόρτα του. Μετά συνδιάζονται όλες οι διαδρομές των δωματίων, ώστε το συνολικό μονοπάτι να είναι βέλτιστο.

Όπως αναφέρεται στο \cite{Galceran2013ASO}, υπάρχουν και άλλες τεχνικές διαχωρισμού του χώρου εκτός του boustrophedon. Υπάρχει ο διαχωρισμός Morse, ο τραπεζοειδής διαχωρισμός, ο διαχωρισμός Morse κάνοντας χρήση και του GVD κ.α.

Οι ευριστικές μέθοδοι χρησιμοποιούν διάφορα μοτίβα για να καλύψουν πλήρως τον χώρο. Oι Faud και Purwanto \cite{fuad2014} χρησιμοποιώντας την εικόνα της κάμερας του ρομπότ το κατευθύνουν να ακουλουθεί τους τοίχους των διαδρόμων (wall following) διατηρώντας από αυτούς μια σταθερή απόσταση. Έτσι, πραγματοποιεί έμμεσα την πλήρη κάλυψη αυτών των χώρων.


