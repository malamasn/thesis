\section{Εξαγωγή Πληροφορίας Δωματίων και Διαδρόμων}
\label{section:room_segmentation}

Ο διαχωρισμός ενός OGM ενός χώρου μπορεί να γίνει με διάφορες προσεγγίσεις όπως προτείνεται από τον Bormann κ.ά. \cite{bormann2016}. Αυτές είναι ο μορφολογικός διαχωρισμός, ο διαχωρισμός με βάση τις αποστάσεις, ο διαχωρισμός εξάγοντας χαρακτηριστικά και ο διαχωρισμός βασιζόμενος στο διάγραμμα Voronoi (Generalized Voronoi Diagram - GVD). Σε κάθε περίπτωση χρησιμοποιούνται διαφορετικά στοιχεία από τον εκάστοτε χάρτη. 

Πιο αναλυτικά, στον μορφολογικό διαχωρισμό ο OGM μετασχηματίζεται σε δυαδική εικόνα και ένας μορφολογικός τελεστής διάβρωσης περνάει επαναληπτικά από αυτήν. Περιοχές που κατά τη διαδικασία αυτή αποκόπτονται από την ενιαία περιοχή αποτελούν διαφορετικά δωμάτια. Στον διαχωρισμό βάση των αποστάσεων υπολογίζεται ένας χάρτης αποστάσεων του κάθε σημείου με το κοντινότερο του και ο OGM μετασχηματίζεται σε δυαδική εικόνα. Στη συνέχεια, επαναληπτικά χρησιμοποιείται ένα κατώφλι απόστασης, το οποίο σταδιακά αυξάνεται. Ο αλγόριθμος συγκρίνει κάθε σημείο με το κατώφλι, κρατάει όσα σημεία είναι μικρότερα του και αναζητά περιοχές οι οποίες αποκόπτονται από την ενιαία περιοχή του ελεύθερου χώρου που αποτελούν τα διάφορα δωμάτια. Ο διαχωρισμός εξάγοντας χαρακτηριστικά προσομοιώνει σε κάθε σημείο του χώρου μια μέτρηση ενός αισθητήρα απόστασης (laser) και ανάλογα με τις μετρήσεις που υπολογίζει κατηγοριοποιεί το σημείο ως τμήμα ενός διαδρόμου ή ενός δωματίου. Τέλος, στην περίπτωση χρήσης του GVD υπολογίζονται κρίσιμα σημεία τα οποία μπορεί να αποτελούν σημεία ύπαρξης πόρτας και, επομένως, διαχωρισμού του χώρου. 

Μια άλλη προσέγγιση προτείνεται από τον Brown \cite{brown2016} και περιλαμβάνει την εύρεση σημείων στένωσης του διδιάστατου χώρου και προσπαθεί να χωρίσει τον χώρο βασιζόμενο στα σημεία αυτά. Tα σημεία αυτά προέρχονται από μια παραλλαγή του GVD, που όμως όλα τα τμήματα του διαγράμματος είναι ευθύγραμμα τμήματα. Έτσι, ξεκινάει από τα σημεία αυτά και απομακρύνεται προσπαθώντας να καλύψει ολόκληρο τον χώρο. Δημιουργούνται, δηλαδή, μέτωπα αναζήτησης από το κέντρο περίπου των δωματίων προς τις άκρες. Κάθε σύγκρουση των μετώπων αυτών σημαίνει ότι συναντώνται διαφορετικά δωμάτια, τα οποία και διαχωρίζονται.

Σε ορισμένες περιπτώσεις ταυτόχρονα με τον διαχωρισμό των δωματίων υλοποιείται και η κατηγοριοποίηση τους. Ο συνήθης διαχωρισμός είναι σε δωμάτια, διαδρόμους και σημεία πόρτας. Τα σημεία των πορτών είναι αυτά που διαχωρίζουν τελικά τον χώρο, ενώ η χρήση των ετικετών \emph{διάδρομος} και \emph{δωμάτιο} δίνει την δυνατότητα σε ένα ρομποτικό όχημα να μπορεί να έχει διαφορετική συμπεριφορά σε κάθε περίπτωση. 

Ο Kaleci κ.ά. \cite{kaleci2015} και ο O. M. Mozos \cite{mozos2008} κατηγοριοποιούν κάθε ελεύθερο σημείο του χώρου σε τρεις κλάσεις, δωμάτιο, διάδρομος και πόρτα. Το ρομπότ κατευθύνεται σε κάθε σημείο του χώρου και καταγράφει μια μέτρηση του LiDAR του και χρησιμοποιεί την μέτρηση ως είσοδο ενός ταξινομητή K-means ή Learning Vector Quantization (LVQ). Ο K-means ταξινομεί με μεγάλη ακρίβεια τα δωμάτια και τους διαδρόμους, ενώ αποτυγχάνει να προβλέψει σωστά τις πόρτες. Αντίθετα, ο LVQ παρουσιάζει μια σημαντική βελτίωση στο ποσοστό εύρεσης σημείων πόρτας, χωρίς να μειώνει σημαντικά τα ποσοστά επιτυχίας των άλλων δύο κλάσεων.

Ο Filliat κ.ά. στο \cite{filliat2012} χρησιμοποιούν μια RGBD κάμερα κατά την εξερεύνηση του χώρου. Χρησιμοποιούν την εικόνα βάθους προκειμένου να διαχωρίσουν τα διάφορα δωμάτια, να ανακαλύψουν τις συσχετίσεις μεταξύ τους και να εντοπίσουν αντικείμενα σε κάθε δωμάτιο. Αυτό επιτυγχάνεται με αλγορίθμους υπολογιστικής όρασης και μηχανικής μάθησης σε συνδιασμό και το αποτέλεσμα είναι ένας υψηλού επιπέδου χάρτης, παρόμοιος με αυτούς που θα δημιουργούσε νοητικά ένας άνθρωπός.


